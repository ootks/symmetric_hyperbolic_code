\documentclass[a4paper]{article}
\usepackage[margin=.8in]{geometry}
\usepackage{graphicx}
\usepackage{fancyhdr, multicol}
\usepackage{float}
\fancyhead[L]{\LARGE{Kevin Shu}}
\fancyhead[C] {\LARGE{Induction and Restriction}}
\fancyhead[R]{\LARGE{}}

\author{Kevin Shu}
\title{Induction and Restriction}

%Set up fancy headers.
\pagestyle{fancy}
\usepackage{amsmath, amsthm, amsfonts, amssymb}
\usepackage[parfill]{parskip}

%Define theorem formatting
\newtheorem{theorem}{Theorem}
\newtheorem{lemma}{Lemma}

% Declare some common abbreviations
\newcommand{\R}{\mathbb{R}}
\newcommand{\C}{\mathbb{C}}
\newcommand{\N}{\mathbb{N}}
\newcommand{\Z}{\mathbb{Z}}
\newcommand{\Q}{\mathbb{Q}}
\newcommand{\E}{\mathbb{E}}
\newcommand{\Hom}{\textbf{\text{Hom}}}
\newcommand{\PP}{\textbf{P}}
\newcommand{\SPACE}{\textbf{SPACE}}
\newcommand{\NP}{\textbf{NP}}
\newcommand{\SAT}{\textbf{SAT}}
\newcommand{\pard}[2]{\frac{\partial #1}{\partial #2}}
\DeclareMathOperator*{\argmin}{arg\,min}
\DeclareMathOperator*{\argmax}{arg\,max}
\newcommand{\st}{{\text{ s.t. }}}
\newcommand{\emean}{{\tilde{e}}}

\begin{document}
Let $\Gamma_{n,d}$ be the vector space of $n$-variate polynomials of the form
\[
    p(x) = \sum_{i=1}^d a_i \emean_1^{d-i} \emean_i.
\]

Such a polynomial can be restricted nicely to the projective line
\[
    \{s\vec{1} + t z : t \in \R\},
\]
where $\vec{1}$ is the all 1's vector, and $z$ is a vector whose first entry is $n-1$, and all other entries are $-1$.

In this case, we see that restriction of the polynomial $p$ to this line gives a linear map
\[
    \phi : \Gamma_{n,d} \rightarrow \R[s, t]_d,
\]
which we'll call the restriction map.

In fact, $\phi$ maps $\Gamma_{n,d}$ into the subspace of $\R[s,t]_d$ of polynomials where the coefficient of $ts^{d-1}$ is 0, that is, it $\phi$ can be thought of as mapping to the vector space
\[
    H_{d, 0} = \{c_1 + \sum_{i=2}^d c_i t^{i}s^{d-i}\}.
\]

We now record some information about what this map in the standard bases and its inverse look like:
\begin{theorem}
    Let $p = \sum_{i=1}^d a_i \emean_1^{d-i} \emean_i$, 
    and let $\phi(p) = q(t,s)$, where 
    \[
        q(t,s) = c_1s^d + \sum_{i=2}^d c_i t^{i}s^{d-i},
    \]
    where $c_1 = \sum_{i=1}^d a_i$, and for all other $i$,
    \[
        \begin{pmatrix}
            c_2\\
            c_3\\
            \dots\\
            c_d
        \end{pmatrix}
        = X
        \begin{pmatrix}
            a_2\\
            a_3\\
            \dots\\
            a_d
        \end{pmatrix},
    \]
    where $X_{ij} = -(i-1)\binom{i}{j}$.

    Let $r(s,t) = \frac{1}{t}\frac{\partial}{\partial t} q$, then 
    \[
        p(x) = -q(1, -1)\tilde{e}_1 - \sum_{i=2}^d \frac{1}{i!}\left(\frac{\partial^{(i-2)}}{\partial t^{(i-2)}}r(1,-1)  \right) \emean_1^{d-i}\emean_i.
    \]
\end{theorem}





\end{document}
